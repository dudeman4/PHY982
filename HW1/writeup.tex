\documentclass[]{scrartcl}

%opening
\title{PHY 982 - HW 1}
\author{Xingze Mao \\ Zachary Matheson \\ Thomas Redpath}
\date{}

\begin{document}

\maketitle

$^{11}$Be has 4 protons and 7 neutrons. According to the shell model, we would expect 2 neutrons in a $1s_{1/2}$ state, 4 in $1p_{3/2}$, and the seventh electron in a $1p_{1/2}$ state; however, experimentally we find that the ground state of $^{11}$Be is a $2s_{1/2}$ state, which is certainly related to the fact that $^{11}$Be is a halo nucleus and not a truly bound state $\leftarrow$ \textit{\textbf{Is it correct to say that halo nuclei are not "truly bound"? And is this a sufficient answer?}}

Slightly above the ground state in energy we find the $1/2^-$ state we otherwise might have expected to be the ground state, and above that is the $5/2^+$ state that follows according to the shell model. After that, things start to diverge more noticeably from shell model predictions, but up to this point, the system behaves almost like one would expect via the shell model except for the ground $1/2^-$ state.

\section*{Method}\nonumber
This is text in the Methods section.\\

\section*{Wavefunction plots}

\begin{figure}

\end{figure}


\section*{Phase Shifts and Resonances}

\end{document}
